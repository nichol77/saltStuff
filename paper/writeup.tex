%\documentclass[doublespacing]{elsart}
\documentclass{elsart}
% if you use PostScript figures in your article
% use the graphics package for simple commands
% \usepackage{graphics}
% or use the graphicx package for more complicated commands
% \usepackage{graphicx}
% or use the epsfig package if you prefer to use the old commands
\usepackage{epsfig}

% The amssymb package provides various useful mathematical symbols
\usepackage{amssymb}

% The lineno packages adds line numbers. Start line numbering with
% \begin{linenumbers}, end it with \end{linenumbers}. Or switch it on
% for the whole article with \linenumbers.
\usepackage{lineno}

%\linenumbers

\begin{document}
  \begin{frontmatter}
    
    \title{Measurements of radio propagation in rock salt for the detection of high-energy neutrinos}
    % use optional labels to link authors explicitly to addresses:
    \author[ucl]{Amy Connolly}, 
    \author[ucla]{Abigail Goodhue}, 
    %\author[ucla]{Abigail Goodhue\corauthref{cor}},
    %\corauth[cor]{Corresponding author.}
    %\ead{agoodhue@physics.ucla.edu}
    \author[uh]{Christian Miki}, 
    \author[ucl]{Ryan Nichol}, 
    \author[ucla]{David Saltzberg}
    
    \address[ucl]{Department of Physics and Astronomy, University College London, Gower Street, London  WC1E 6BT}
    \address[ucla]{Department of Physics and Astronomy, UCLA, 475 Portola Plaza, Mailstop 154705, Los Angeles, CA 90095-1547, USA}
    \address[uh]{Department of Physics and Astronomy, University of Hawaii, 2505 Correa Rd., Honolulu, HI  96822, USA}

    \begin{abstract}
      %Text of abstract
    \end{abstract}
    
    \begin{keyword}
      % keywords here, in the form: keyword \sep keyword
      
      % PACS codes here, in the form: \PACS code \sep code
      \PACS 
    \end{keyword}
  \end{frontmatter}
  
  % main text
  \section{Introduction}
  The observation of cosmic rays with energy higher than the Greisen-Zatsepin-Kuzmin~(GZK) 
  cutoff at $~10^{19.5}$~eV~\cite{gzk} implies a corresponding flux of ultra-high energy (UHE) neutrinos, 
  with energy in the $10^{17}-10^{19}$~eV range~\cite{ess}.  These secondary neutrinos 
  are created via photomeson 
  production of cosmic rays on the 2.7~K cosmic microwave background.  Detection of these neutrinos 
  would provide unique information about the origin of primary cosmic rays and the nature 
  of their sources.
  
  Although predictions for the flux of UHE
  neutrinos differ by orders of magnitude, a reasonable set of parameters
  puts the rate of UHE neutrinos at the level of 10/km$^2$/century~\cite{ess}.
  A detector volume of hundreds of cubic kilometers 
  water equivalent is required to detect a significant 
  number of neutrinos in this energy regime.  The largest neutrino detectors 
  currently in operation rely on 
  optical techniques, but the volume that can be instrumented is 
  constrained by attenuation lengths of tens of meters %100 m? says David.  Look up 
  at optical frequencies in detector media (such as ice), so 
  optical detectors have limited sensitivity to rare high-energy events.
  
  However, a few materials that occur naturally in very large volumes are expected to have 
  long attenuation lengths in the radio regime, and are 
  viable as a UHE neutrino detector.  Radio attenuation 
  lengths longer 
  than 1~km have been measured in ice near the South Pole~\cite{southpoleice}, and rock salt 
  formations are thought to have low loss as well.
  G.~Askaryan first predicted coherent radio emission from ultra 
  high-energy showers
  ~\cite{askaryan}, and the effect has been confirmed in 
  accelerator beam tests~\cite{sandsaltice}.
  The energy of the coherent radio emission that results from the development 
  of a negative charge excess in the shower depends quadratically on shower energy, and for 
  showers with energy greater than $10^{16}$~eV, dominates 
  the emitted Cherenkov power spectrum.  
  
  Formations of salt rock could therefore be a viable detector for high-energy neutrinos if 
  attenuation in the radio regime is indeed low.  
  Domes of rock salt occur naturally in many parts of the world, including 
  the Gulf Coast region of the United States.  The salt originates from
  dried ocean beds which have been buoyed upward due to geological forces
  through a process called diapirism.  Through this process, the salt
  becomes very pure as impurities are extruded.
  Such salt domes, with typical dimensions of several square kilometers by several kilometers deep, 
  are thought to be good candidates for a neutrino detector.  
  For a more detailed discussion of the dielectric properties of salt and the application to neutrino 
  detection, see reference~\cite{hockley}.

  \section{Previous measurements}
  There are two classifications of measurements that have been made of the radio properties 
  of rock salt in salt domes.  Ground Penetrating Radar~(GPR) was used in the 1960's and 
  1970's to determine the size and structure of salt domes.  This technique is based on sending 
  a radar signal into the salt and 
  looking for reflections from interfaces in the salt structure.  One can calculate the 
  distance of the interface from the time delay of the reflected pulse.  Although GPR measurements 
  were taken to calculate attenuation losses in the salt, one can also extract an attenuation 
  coefficient based on the detected signal voltage, the voltage of the transmitted pulse, and the distance 
  over which the pulse traveled.  The GPR technique is limited because of unknown reflection 
  coefficient at the surface of reflection, but assuming a coefficient of $1.0$ gives a conservative estimate.

  Direct measurements of attenuation in rock salt have also been made.  In 2002, measurements made at 
  Hockley salt mine showed attenuation lengths consistent with being longer than 40~m~\cite{hockley}.  
  The setup that was used was limited by the voltage of the pulser, and could not transmit over 
  very long distances through salt.  We decided 
  to follow the techniques of the direct transmission measurements but used a high voltage pulser 
  so that we could transmit through longer distances of salt.  
  
  \section{Cote Blanche mine experiments}
  We made measurements of the dielectric properties of salt in the mine located in the Cote Blanche 
  salt dome in St.~Mary Parish, Louisiana in August 2007.  
  We chose this dome because GPR measurements~\cite{s-unterberger} that were made in the mine
  suggested very low radio attenuation, seeing reflection over the longest 
  distance of any mine measured.  
  
  The Cote Blanche dome is one of five 
  salt domes in the area.  The salt dome extends from approximately 90~m below the 
  surface to 4270~m below the surface.  At a depth of 1100~ft.~(335~m), the salt 
  extends 1700~m east to west and 2100~m north to south.  At 2000~ft.~(610~m) deep, 
  the horizontal cross section has nearly doubled in size.   The total volume of 
  salt in the dome is estimated to be 28-30~km$^3$~\cite{halbouty, cherryelog, davidw}.
  
  The dome has been actively mined since~1965 using the 
  conventional room and pillar method.  The mine consists of a grid of 30~ft.~(9~m) 
  wide by 30~ft.~(9~m) high drifts (hallways) spaced by 100~ft. by 100~ft. (30~m by 30~m)
  pillars of salt, and has three levels at depths of 1100~ft., 1300~ft., 
  and 1500~ft.~(335~m, 396~m, and 457~m).  Each level covers 
  several square kilometers.  Current mining operations are on the 1500~ft. deep level.  
  
  The measurements described in this paper are from the third trip that we made to the Cote 
  Blanche mine.  In May 2005, our first visit to the site, with 
  M. Cherry and J. Marsh from Louisiana State University, we 
  established the viability of the experimental setup\cite{firstvisit}.  
  In September 2006, on a return visit to the mine, we transmitted and 
  received signals horizontally across a 
  single pillar of salt, both with antennas placed against the walls 
  and with antennas 4~m into the ceiling in shallow boreholes.  
  We measured attenuation lengths 
  of $24.6\pm2.2$~m in frequency range 50-150~MHz, 
  $22.2\pm1.8$~m at 150-250~MHz, 
  and $20.5\pm1.5$~m 
  at 250-350~MHz.  We also measured an average index of 
  refraction of $n=2.4\pm0.1$~\cite{secondvisit}.  
  
  The relatively short attenuation lengths measured through the walls of the 
  pillar could be due to the method used to mine the salt. 
  The miners carve out corridors in the salt by cutting a 
  horizontal slice out from under the wall and then blasting the 
  section above the floor so that the salt can be removed, and
  then procede to a new section of salt further along the corridor.
  The data from our first two trips were consistent with a model of
  very lossy salt (approximately 10~m attenuation length) 
  in the region closest
  to the walls of the pillar  
  and longer attenuation lengths (quoted above) 
  in salt more than about 10~m from the wall.  
  
  \subsection{Experimental setup}
  
  Figure~\ref{fig:saltmap} shows the region of the 1500~ft. deep level of the mine 
  where we made our measurements, and Figure~\ref{fig:systemdiagram} is 
  a diagram of the experimental.  
  The miners drilled three holes with a 2.75~inch (7.0~cm) diameter into the salt 
  beginning at the 1500~ft. level of the mine.
  Boreholes~1 and~2 were 100~ft. deep, while Borehold~3 was 200~ft. deep.
  At the first two boreholes, drilling was stopped at 100~ft. once the
  driller encountered methane gas.
  We used a fast, high-power, broadband Pockel's cell 
  pulser from Grand Applied Physics with a peak voltage of 2500~kV and a 
  10\%-90\%~rise time of 200~ps to generate radio pulses.  
  There was 5~ft. of LMR~240 cable and
  200~ft. of LMR~600 cable leading to the transmitting as well as from the 
  receiving antennas, and the antennas were lowered by hand into the salt.
  
  We made most of our measurements using three pairs 
  of dipole antennas.  We measured the 3~dB points of each antenna when they were  
  in a borehole in the salt to determine the in-band 
  frequencies.  The 3~dB points of the transmission
  of the low frequency (LF) antenna pair (Raven Research RR6335), 
  are 50~to~175~MHz in salt.  For the medium frequency (MF) antennas, which we custom made, 
  the 3~dB points are
  175~and~500~MHz in salt.  The high frequency (HF) pair 
  (Shure Incorporated UA820A), have a transmission 
  band between 550~and~900~MHz in salt.
  
  We used a Tektronix TDS694C oscilloscope to 
  record the received signal pulse.  We used a pulser with syncronized outputs
  to trigger both the the high-power pulser and the oscilloscope.  This allowed
  us to look in a fixed time window for the signal and reduce noise by averaging
  many waveforms.
  
  The transmitting antenna was in the borehole at 
  Location~1, and the identical receiving antenna was either in hole~2, 50~m away from the 
  transmitter, or hole~3, 169~m away from the transmitter.  We took data at 10~ft.~(3~m) 
  incremental depths in the boreholes.  The setup used to measure transmission between 
  Boreholes~1~and~2 was identical to the setup used to 
  measure between Boreholes~1~and~3 
  so that we could make reliable relative measurements without requiring absolute system 
  calibration.
  Because the transmission band of each antenna was relatively broad, we were able to make 
  measurements between 100~and~900~MHz in salt. 
  


  Using the same pulser that was used for the attenuation measurements,
  we measured the fraction of power reflected from each antenna while it
  was in a borehole  
  so that we could deduce the fraction transmitted and it frequency dependence. 
  For this S11~measurement, we used the standard method of inserting
  a coupler (Minicircuits ZFDC-20-4) between the pulser and antenna and 
  recording the reflected signal through the coupled port.
  After measuring the reflection
  from the open cable (with the setup identical but with the antenna only
  removed), then the ratio of the power in the 
  two pulses is the fraction reflected from the antenna.  The transmitted
  fraction is deduced by taking the sum of the 
  reflected and transmitted power fractions to be unity. 
  Figure~\ref{fig:s21} shows the transmission 
  of a medium frequency antenna as a function of frequency.  The transmission 
  did not change 
  significantly when we changed the depth of the antenna in the hole.  
  
  \begin{figure}
    \epsfig{figure=saltmap.eps, height=4.0in}
    \caption{Map showing section of 1500 ft. level of the mine where we made our measurements.}
    \label{fig:saltmap}
  \end{figure}
  \begin{figure}
    \epsfig{figure=systemdiagram.eps, height=3.5in}
    \caption{System diagram.}
    \label{fig:systemdiagram}
  \end{figure}
  \begin{figure}
    \epsfig{figure=S11ChristianSalty_BW.eps, height=3.5in}
    \caption{Measured transmission for the medium frequency antennas.}
    \label{fig:s21}
  \end{figure}
  

  \subsection{Attenuation measurements}
Since the system we used was the same regardless of the location of the 
  receiver, the received voltages at Boreholes~1 and~2 ($V_{12}$ and $V_{13}$) at
measured distances ($d_{12}$ and $d_{13}$) from the transmitter is given by:
  \begin{equation}
   \frac{ V_{13}}{V_{12}} = \frac{d_{12}}{d_{13}} \cdot \exp(-(d_{13}-d_{12})/L_\alpha)
  \end{equation}
  where~$L_\alpha$ is the field attenuation length.  Inverting this equation givs an expression for the field attenuation length: 
  \begin{equation}
    \label{eq:atten}
    L_\alpha=(d_{13}-d_{12})/\ln( \frac{d_{12}V_{12}}{d_{13}V_{13}} )
  \end{equation}
  
  Figure~\ref{fig:timedomain12_13} shows an example waveform 
  of the received pulse at Borehole~2 and Borehole~3, and Figure~\ref{fig:freqdomain12_13} 
  shows the Fourier transform of the same waveforms. To calculate the attenuation length, we 
  first cut the recorded waveforms in a time window around the pulse to 
  eliminate any reflections and reduce contributions from noise.  The width 
  of the window is 35~ns for the low frequency antennas, 30~ns for the medium frequency antennas, 
  and 12~ns for the high frequency antennas.  

  We expect to record reflections from interfaces within the salt and between the salt and air.  
  Because of the geometry of our setup, we could see reflections from the corridor above 
  the antennas, and can calculate the relative time when this reflection would appear as a function 
  of depth of the antennas in the boreholes.  The time window that we use to for the analysis 
  extends no more than 20~ns after 
  the peak of the pulse to ensure that we eliminate the possibility of interference from 
  any reflection off of the corridor for depths of 50~ft. and below.  
  
  Using the Fourier transform, we sum the 
  total power in a frequency band.  We use a 50~MHz wide band for the low frequency and 
  medium frequency antennas, and a 100~MHz wide band for the 
  high frequency antennas which had a larger bandwidth.  We do a simple noise subtraction by taking the Fourier transform of 
  a time window of the same length of time as the pulse 
  window where no pulse was present and subtracting the power contributions in band.  
  We then calculate the attenuation length using equation~\ref{eq:atten}.  An estimation 
  of uncertainties is discussed in section~\ref{sec:error}.
  
  \begin{figure}
    \epsfig{figure=time_12_13.eps, height=4.0in}
    \caption{Pulses that propagated directly between the transmitter and receiver between Boreholes~1~and~2 between Boresholes~1~and~3 at a depth of 90~ft.}
    \label{fig:timedomain12_13}
  \end{figure}
  
  \begin{figure}
    \epsfig{figure=freq_12_13.eps, height=4.0in}
    \caption{Fourier transforms of pulses in Figure~\ref{fig:timedomain12_13}.}
    \label{fig:freqdomain12_13}
  \end{figure}
  
  Figure~\ref{fig:depth} shows the attenuation length at 250~MHz as measured with the 
  medium frequency antennas as a function of the depth of the pair of antennas in the hole.  
  The attenuation 
  length of the salt changes at different depths, but the amount of loss is not 
  a function of increasing depth.    
  We took measurements at 10~ft. and 20~ft. depths, but since the deeper measurements 
  are more likely to be of unfractured, clear salt, and do not interfere with reflections, 
  we only show the results from 30~ft. and deeper.  
  The attenuation lengths measured at 10~and~20~ft. depths are generally consistent 
  with the deeper measurements.
  
  \begin{figure}
    \epsfig{figure=depth.eps, height=4.0in}
    \caption{Attenuation length at 250~MHz measured with the medium frequency antennas.}
    \label{fig:depth}
  \end{figure}
  
  Measured field attenuation length at 50~ft. and 90~ft. depths 
  is shown in Figure~\ref{fig:freq} as a function of 
  frequency.  If the salt had a constant loss tangent, we expect that 
  the attenuation length would decrease with 
  increasing frequency as $\nu^{-1}$, where $\nu$ is the frequency of the radiation~\cite{hockley}.    
  A power law fit to the data gives a spectral index of $-0.505\pm0.004$ with a normalization 
  constant of $39.8\pm0.25$.  Although it is clear that
  the attenuation length is falling with increasing frequency, it does not follow the $\nu^{-1}$ 
  expectation, which hints at a non-constant loss tangent, 
  consistent with measurements at the Hockley 
  mine~\cite{hockley}.  The attenuation length is as high as 130~m at 125~MHz, 
  and as low as 30~m at 900~MHz.
  %chisquare/dof is 9367/34.  What?

  We have also averaged the attenuation length at each frequency over all depths 
  of 50~ft. and deeper and over all antenna types, 
  and plotted the results in Figure~\ref{fig:avg}.  We show a power law fit to the data with a spectral 
  index of $-0.65\pm0.09$ and a normalization constant of $34.7\pm3.8$.  
  The uncertainties shown in the figure are based on the 
  scatter between measured attenuation lengths using different antennas in the same frequency bin.
  %chisquare/dof is 6.15/12 .  How do you want to quote this.
  
  The attenuation lengths that we measured on this trip are significantly longer than those 
  that we measured on our previous trip to the same mine.  The measurement that we report here 
  was made in holes that were drilled into the floor of the lowest level of the mine, whereas 
  the previous measurements were made with holes drilled into the ceiling.  Because the method  
  of mining conisists of cutting under the salt and then blasting the wall above the cut, any 
  fractures that occur due to the process would tend to propagate upward.  This means that the salt 
  in the floor of the lowest level of the mine would tend to be less fractured and therefore 
  exhibit less loss in the radio regime.  

  \begin{figure}
    \epsfig{figure=freq.eps, height=4.0in}
    \caption{Attenuation length shown as a function of frequency for two depths and all three pairs of dipoles.}
    \label{fig:freq}
  \end{figure}
  \begin{figure}
    \epsfig{figure=avg.eps, height=4.0in}
    \caption{Average attenuation length, shown as a function of frequency.}
    \label{fig:avg}
  \end{figure}
  
  \subsubsection{Uncertainty estimation}
  \label{sec:error}
  The main source of systematic uncertainty on the measurement of 
  the field attenuation length at a given depth 
  is due to the position of the antenna within the hole.  
  In previous trips to Cote Blanche, we discovered that a small variation in position of the antenna 
  led to a significant change in the voltage received through the salt~\cite{secondvisit}.  
  We estimate the size of this uncertainty as the typical variation between neighboring depths of 
  the peak-to-peak voltage of the recorded waveform at each depth
  (excluding the 10~and~20~ft. depths), or
  
  \begin{equation}
    \delta V = \frac{1}{N} \sqrt{\sum_{i=0}^{N-1} (V_{i}-V_{i+1})^2} 
  \end{equation}
  where $N$ is the number of depths included here. Using this method, 
  we estimate the uncertainty on the voltage measured due to the position of the antenna 
  in the hole to be $24\%$.
  
  Another source of uncertainty is the sample-to-sample variations observed when the 
  antennas were at the same location.  The typical variation observed between samples of the same 
  setup was a voltage variation of $5.4\%$.
  
  We also include an uncertainty due to the exact choice of the time window that contains the pulse.  
  We estimate this uncertainty as the root mean square variation 
  of the total power in the pulse as we slide 
  the time window by 1~ns.  When the power at a given frequency is small, 
  this uncertainty dominates (up to 50\% in voltage), but 
  in the frequency band of the antenna, the uncertainty is small (less than 10\% in voltage).
  
  There is also an uncertainty due to how well we know the distance between the holes.  We 
  measured the distance between holes~1~and~2 with a measuring tape, and then used relative 
  timing of received pulses to extrapolate the distance between holes~1~and~3.  
  Combining an uncertainty due to unknown 
  timing in the system, one due to the measurement of the distance through, and one  
  due to the fact that the holes may not be drilled straight down, 
  we calculate that the maximum uncertainty on the distance between the transmitting and receiving 
  antennas is always less than $2.1\%$.

  \subsection{Index of refraction measurements}
  We also calculate the index of refraction of the salt using the direct transmission
  measurements that we made.  The index of refraction is defined as $n=\sqrt{\epsilon^\prime}$, 
  where $\epsilon^\prime$ is the real part of the dielectric permittivity.  We calculate 
  the speed of transmission through the medium using the distance between holes~1~and~2 and 
  the time of travel of the signal pulse through the salt.  The time of travel was measured 
  by taking the difference between the time that the pulse was generated and the signal was 
  received and subtracting the known system delay.
  
  Figure~\ref{fig:n} shows the measured index of refraction at each depth.  The index of refraction 
  that we measure is consistent with $n=2.45$, the index of refraction of rock salt.  We estimate 
  an uncertainty of 3~ns on the absolute system timing and 1~ft. on the distance between 
  the holes.  We also include an uncertainty due to the fact that the holes may not be 
  drilled straight down, using a depth dependent wander that has a maxiumum of 3~ft. 
  at the 100~ft. depth.
  
  \begin{figure}
    \epsfig{figure=n.eps, height=4.0in}
    \caption{Index of refraction measured at each depth.}
    \label{fig:n}
  \end{figure}

  \subsection{Reflection measurements}
  
  We expected that signals transmitted from within the salt
  would reflect from nearby surfaces and be observable
  in our data, and indeed at nearly every measurement position we observed
  at least one clear ``reflected'' signal in addition to the 
  ``direct'' signal that traveled along the straight line between 
  the transmitting and receiving antennas.
  Reflected signals can be easily distinguished from direct signals
  from their time of arrival at the receiver.  Reflected signals allow
  us to probe distances greater than that between our drilled holes,
  and to probe different salt regions as well.  However,
  little is known about the loss in power incurred at the reflection, and
  reflected signals are often transmitted and received by the antennas at
  oblique angles, where the antenna response is less well understood.
  
  With our antennas in Boreholes~1~and~2, the shortest path between the
  receiver and transmitter was below the corridor
  where we were working, so we expected to
  see reflections from that corridor.  We did observe signals that
  were consistent with this interpretation.
  The meausured time differences
  between the direct and reflected signals while we were transmitting
  between Boreholes~1~and~2
  were within approximately 10~ns of the expected time
  difference at all depths.  These reflected pulses traversed as
  much as 256~ft. (78~m) of salt, and a discrepancy of 10~ns corresponds to
  approximately 4~ft. (1.2~m) in salt.  These reflected pulses would have been
  transmitted from and received by the antennas at angles relative to 
  the horizontal that increased with depth, as high as 48$^{\circ}$ when the
  antennas were 90~ft. deep.

  As the antennas were lowered to increasing depths in Boreholes~1~and~2, 
  we observed the 
  magnitude of the reflected signal {\em increase} and the pulse shape 
  become increasingly narrow in time.  
  % This same behavior was observed for all three antennas used at
  % these measurement positions, the HF, LF and Hawaii antennas.
  At 90~ft. depth, the peak-to-peak voltage of this reflected pulse was
  x times that of the direct signal.

  This increase in the signal strength with increased depth (and hence
  increased angle relative to horizontal) was
  constrained to frequencies below the bandwidth measured on-beam  
  for each antenna.  This behavior is consistent with the interpretation that
  we were either seeing a side-lobe effect of the antennas at low frequencies,
  or that the antennas, or the antennas combined with a section of the
  cables, were acting as long-wire antennas.  The beam pattern of long-wire antennas 
  (for antennas lengths that are odd multiples of the half-wavelength) have a null
  at angles perpendicular to the wire axis, with lobes at increased angles.
  Since we did not observe any additional nulls as the antennas were lowered
  in the boreholes, and since the timing of the pulses is consistent with
  their source and being the antennas, if the long-wire antenna interpretation 
  is correct, the relevant length of cable contributing to this ``antenna'' is not a 
  large number of wavelengths. 
  
  %The dependence of the signal strength on angle of signal propagation relative to
  %the horizontal
 
%  \subsection{S12 in Salt}
  
\subsection{Beam pattern in salt}

\begin{figure}
  \epsfig{figure=beampattern.eps, height=4.0in}
  \caption{Measurement of beam pattern while one medium frequency antenna was at 90 ft. depth in Borehole~1 and the other was at various depths in Borehole~2~or~3.}
  \label{fig:beampattern}
\end{figure}

\section{Review of GPR data}

GPR measurements were made in the Cote Blanche mine and other salt mines in the 
1970's by Stewart and Unterberger ~\cite{unterberger, s-unterberger} to probe 
for discontinuities in the salt.  They used a single frequency waveform generator, a pair of 
high-gain antennas that were pointed into the salt of interest, and an oscilloscope to measure the time delay 
between the transmitted signal and any received reflections off of discontinuities deep within salt.  
To calculate the distance of the discontinuities, they used an index of refraction measured via 
transmission across a pillar of salt, a technique similar to our method.  

The anntennas were used to measure the location of the top of the Cote Blanche dome from within 
the mine by transmitting the signal vertically through the ceiling.  At one measurement station, 
a multiply-reflected signal that had traveled through a total pathlength of 4080~ft.~(1244~m) 
was observed.  This is the longest transmission distance observed in any mine.  
We have compiled information in Table~\ref{tab:gprspecs} 
from the discussion of the specification of the system used to make the measurements in 
references~\cite{unterberger}~and~\cite{s-unterberger}.

The power received, $P_{Rx}$, is related to the power transmitted, $P_{Tx}$, by the Friis formula:
\begin{equation}
  P_{Rx}=P_{Tx}\frac{G_{Tx} G_{Rx} \lambda^2}{(4\pi r)^2}
\end{equation}
where $G_{Tx}$ and $G_{Rx}$ are the gain of the transmitting and 
receiving antennas, $\lambda$ is the wavelength of 
the transmitted signal in salt, and $r$ is the distance between 
the antennas.  Using the result of the Friis 
formula togehter with the GPR system specifications, 
we calculate that the minimum attenuation length necessary to detect the reflected signal 
over the longest distance observed (see Table~\ref{tab:gprspecs}) is 138~m.  

This attenuation length is not inconsistent with the results we report in this paper, 
although our results show generally higher losses.
Because we were only able to sample one section of salt with our measurements, we cannot reasonably 
compare our results with the lowest-loss GPR measurements.  In order to replicate the low-loss results 
of the GPR technique, any further measurements would have to be made with a portable 
high-power system capable of making measurements at many locations.  

We made a first attempt at making measurements using the GPR technique during our visit to 
the mine.  Using the same high-power pulser and oscilloscope, we used a pair of directional 
antennas in an attempt to see reflections off of interfaces within the walls of the mine.  We did 
see reflected pulses, but were unable to understand the source of the reflections.  We did not 
understand the behavior of the system well enough to conclude whether the reflections that we 
saw came from within the salt, or were merely reflections down the corridor of the mine.  

\begin{table}
  \begin{center}
    \begin{tabular}{lc}
      \multicolumn{2}{c}{} \\ \hline\hline
      System Specifications &\\\hline
      Peak Power Output & 40 dBm (10 W)\\
      Antenna Gain & 17 dB\\
      Frequency & 440 MHz\\
      Maximum Transmission Distance & 1244 m\\
      System Sensitivity & -100 dBm ($10^{-13}$ W)\\\hline\hline
      Attenuation Length Calculation & \\\hline
      Loss from transmission (from Friis formula) & 61 dB\\
      Power received (assuming no attenuation) & -21 dBm ($8.0\times 10^{-6}$ W)\\
      Maximum attenuation allowed & -79 dB (a factor of $1.3\times 10^{-8}$)\\
      Attenuation length & 138 m\\\hline
    \end{tabular}
  \end{center}
  \caption{Specifications of the GPR system used in~\cite{s-unterberger}~and~\cite{unterberger}, 
    and a calculation of the minimum attenuation length needed to see the longest pathlength 
    reflection observed.}
  \label{tab:gprspecs}
\end{table}



\section{Simulation}


\section{Conclusions}


  
% The Appendices part is started with the command \appendix;
% appendix sections are then done as normal sections
% \appendix

% \section{}
% \label{}

\begin{thebibliography}{00}
  
  % \bibitem{label}
  % Text of bibliographic item
    
    % notes:
    % \bibitem{label} \note
    
    % subbibitems:
    % \begin{subbibitems}{label}
    % \bibitem{label1}
    % \bibitem{label2}
    % If there is a note, it should come last:
    % \bibitem{label3} \note
    % \end{subbibitems}
    
\bibitem{gzk}
  G.~T.~Zatsepin and V.~A.~Kuzmin,
  JETP Lett.\  {\bf 4}, 78 (1966)
  [Pisma Zh.\ Eksp.\ Teor.\ Fiz.\  {\bf 4}, 114 (1966)];
  %%CITATION = JTPLA,4,78;%%
  %%Cited 644 times in SPIRES-HEP
%\cite{Greisen:1966jv}
  K.~Greisen,
  Phys.\ Rev.\ Lett.\  {\bf 16}, 748 (1966).
  %%CITATION = PRLTA,16,748;%%
  %%Cited 951 time in SPIRES-HEP

\bibitem{ess} R. Engel {\it et al.}, Phys.Rev. D64:093010.
\bibitem{southpoleice}S.W.Barwick, {\em et al.} 2005 J.Glaciology 51 231.
\bibitem{askaryan}G. Askaryan, Soviet Physics JETP-USSR 14 (2): 441-443 1962.
\bibitem{sandsaltice}D. Saltzberg {\em et al.} Phys. Rev. Lett. 86:2802-2805,2001; 
  P.W.Gorham {\em et al.} Phys. Rev. D 72:023002,2005; 
  P.W.Gorham {\em et al.} Phys. Rev. Lett. 99:171101,2007.
\bibitem{hockley}P. Gorham, D. Saltzberg {\em et al.}, Nucl.Instrum.Meth.{\bf A}490:476-491,2002.
\bibitem{halbouty}M. Halbouty, Salt Domes: Gulf Region, United States and Mexico, Gulf Publishing Co., Houston, TX, 1967.
\bibitem{davidw} D. W.
\bibitem{cherryelog}M. Cherry
\bibitem{s-unterberger}Stewart and Unterberger, Geophysics 41 (1):123-132, 1976.
\bibitem{unterberger}Unterberger, Geophysical Prospecting 26 (2):312-328, 1978.
\bibitem{firstvisit}A. Connolly, A. Goodhue, M. Cherry, and J. Marsh
\bibitem{secondvisit}A. Connolly, A. Goodhue, D. Saltzberg
  
\end{thebibliography}

\end{document}

