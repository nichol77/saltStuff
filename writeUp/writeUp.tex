%\documentclass[doublespacing]{elsart}
\documentclass{article}
% if you use PostScript figures in your article
% use the graphics package for simple commands
% \usepackage{graphics}
% or use the graphicx package for more complicated commands
% \usepackage{graphicx}
% or use the epsfig package if you prefer to use the old commands
\usepackage{epsfig}

% The amssymb package provides various useful mathematical symbols
\usepackage{amssymb}

% The lineno packages adds line numbers. Start line numbering with
% \begin{linenumbers}, end it with \end{linenumbers}. Or switch it on
% for the whole article with \linenumbers.
\usepackage{lineno}

%\linenumbers

\begin{document}
    
    \title{Measurements of radio propagation in rock salt for the detection of high-energy neutrinos}
    % use optional labels to link authors explicitly to addresses:
    \author[ucl]{Amy Connolly}, 
    \author[ucla]{Abigail Goodhue}, 
    %\author[ucla]{Abigail Goodhue\corauthref{cor}},
    %\corauth[cor]{Corresponding author.}
    %\ead{agoodhue@physics.ucla.edu}
    \author[ucl]{Ryan Nichol}, 
    \author[ucla]{David Saltzberg}
    
    \address[ucl]{University College London}
    \address[ucla]{Department of Physics and Astronomy, UCLA, 475 Portola Plaza, Mailstop 154705, Los Angeles, CA 90095-1547, USA}
      
    \begin{abstract}
      % Text of abstract
    \end{abstract}
    
    \begin{keyword}
      % keywords here, in the form: keyword \sep keyword
      
      % PACS codes here, in the form: \PACS code \sep code
      \PACS 
    \end{keyword}
\maketitle
  
  % main text
  \section{Introduction}
  The observation of cosmic rays with energy higher than the Greisen-Zatsepin-Kuzmin~(GZK) 
  cutoff at $~10^{19.5}$~eV~\cite{gzk} implies a corresponding flux of neutrinos 
  with energy of $10^{17}-10^{19}$~eV~\cite{ess}.  These secondary neutrinos 
  are created via photomeson 
  production of cosmic rays on the 3~K cosmic microwave background.  Detection of these neutrinos 
  would provide unique information about the origin of primary cosmic rays and the nature 
  of their sources.
  
  The predicted flux of these 
  ultra high-energy neutrinos is small, requiring a very large detector 
  volume of hundreds of cubic kilometers water equivalent to detect a significant 
  number of neutrinos in this energy regime.  Most neutrino detectors currently rely on 
  optical techniques, but the volume that can be instrumented is limited by relatively high 
  attenuation of optical frequencies in detector materials (such as ice), so 
  optical detectors are not sensitive to rare high-energy events.
  
  However, naturally occuring media with long attenuation lengths in the radio regime are 
  viable as a detector.  It is known that ice is very radio-clear, with a radio attenuation 
  length longer 
  than 1~km~\cite{Barwick}, and rock salt formations are are thought to have low loss as well.
  Askaryan first predicted coherent radio emission from ultra high-energy showers
  ~\cite{askaryan}, and the effect has been confirmed in experiments at SLAC~\cite{saltzberg}.
  The energy of the coherent radio emission that results from the development 
  of a negative charge excess in the shower depends quadratically on shower energy, and for 
  showers with energy greater than $10^{16}$~eV dominates the emitted Cherenkov power spectrum.  
  
  Formations of salt rock could therefore be a viable detector for high-energy neutrinos if 
  attenuation in the radio regime is low.  Domes of rock salt occur naturally in the Gulf Coast 
  region of the United States.  During the formation of such domes, the salt becomes very pure 
  as it is pushed up from the salt bed below in a process called diapirism.  Therefore, salt 
  domes are thought to be good candidates for a neutrino detector location.  For a more 
  detailed discussion of the dielectric properties of salt and the application to neutrino 
  detection, see Gorham and Saltzberg et. al.~\cite{hockley}.

  \section{Previous measurements}
  There are two classifications of measurements that have been made of the radio properties 
  of rock salt in salt domes.  Ground Penetrating Radar~(GPR) was used in the 1960's and 
  1970's to determine the size and structure of salt domes.  This technique is based on sending 
  a radar signal into the salt and 
  looking for reflections off of interfaces in the salt structure.  One can calculate the 
  distance of the interface from the time delay of the reflected pulse.  Although these measurements 
  were not focused on calculating attenuation losses in the salt, one can try to extract an attenuation 
  coefficient based on an assumed noise floor, the voltage of the transmitted pulse, and the distance 
  over which the pulse travelled.  The GPR technique is limited because of unknown reflection 
  coefficient at the surface of reflection.

  Direct measurements of attenuation in rock salt have also been made.  In 2002, measurements made at 
  Hockley salt mine gave attenuation lengths consistent with being longer than 40~m~\cite{hockley}.  
  The setup that was used was limited by the voltage of the pulser, and could not transmit over 
  very long distances through salt, leaving large uncertainties on the results.  We decided 
  to follow the techniques of the direct transmission measurements but used a high voltage pulser 
  so that we could transmit through longer distances of salt.  
  
  \section{Cote Blanche mine experiments}
  We decided to make our measurements in the Cote Blanche salt dome in St. Mary Parish, Louisiana.  
  We chose this dome because GPR measurements that were made in the mine
  hinted at very low radio attenuation, seeing reflection over the longest 
  distance of any mine measured~\cite{unterberger}.  
  
  The Cote Blanche dome is one of five 
  salt domes in the area.  The salt dome extends from approximately 90~m below the 
  surface to 4270~m below the surface.  At a depth of 1100~ft.~(335~m), the salt 
  extends 1700~m east to west and 2100~m north to south.  At 2000~ft.~(610~m) deep, 
  the horizontal cross section has nearly doubled in size.   The total volume of 
  salt in the dome is estimated to be 28-30~km$^3$~\cite{halbouty},\cite{cherryelog}.
  
  The dome has been actively mined since~1965 using the 
  conventional room and pillar method.  The mine consists of a grid of 30~ft.~(9~m) 
  wide by 30~ft.~(9~m) high drifts (hallways) spaced by 100~ft. by 100~ft. (30~m by 30~m)
  pillars of salt, and has three levels at depths of 1100~ft., 1300~ft., 
  and 1500~ft.~(335~m, 396~m, and 457~m).  Each level covers 
  several square kilometers.  Current mining operations are on the 460~m deep level.  
  
  We made measurements of the dielectric properties of salt in the Cote Blanche mine 
  in August 2007.  We used a fast, high-power, broadband pulser 
  to generate radio pulses.  We made most of our measurements using three pairs 
  of dipole antennas whose transmission peaked at 200~MHz, 400~MHz, and 800~MHz in air.  
  Figure~\ref{fig:s21} shows the transmission coefficient of each dipole antenna when it was 
  placed in a borehole in the salt.  The 
  transmission coefficient did not change significantly when changed the depth that the antenna 
  was at in the hole.  
  We placed the antennas in boreholes that were drilled as deep as 200~ft.~(61~m) into the floor 
  of a drift, and transmitted radio signals through salt between pairs of identical antennas.
  Because the transmission band of each antenna was relatively broad, we were able to make 
  measurements between 100~and~1000~MHz in salt. 
  The measurements described in this paper are from the thrid trip that we made to the Cote 
  Blanche mine.  In May 2005, a preliminary 
  scouting visit to the site proved that the experimental setup would work~\cite{firstvisit}.  
  In September 2006, on a return visit to the mine, we made attenuation length 
  and index of refraction measurements by transmitting horizontally across a 
  single pillar of salt both with antennas placed against the walls 
  and with antennas 4~m deep in shallow boreholes.  We measured an attenuation length 
  of $24.6\pm2.2$~m for 50-150~MHz, $22.2\pm1.8$~m for 150-250~MHz, and $20.5\pm1.5$~m 
  for 250-350~MHz, and an average index of refraction $n=2.4\pm0.1$.~\cite{secondvisit}.  
  The relatively short attenuation lengths measured through the walls of the pillar 
  may indicate where explosives are used to mine material, in general 
  there is more attenuation in the locally disturbed salt, for example near drifts, 
  than in the bulk.
  
  Figure~\ref{fig:map} shows the region of the 1500~ft. deep level of the mine 
  where we made our measurements.  The transmitting antenna was in the borehole at 
  location~A, and the receiving antenna was either in hole~B, 50~m away from the 
  transmitter, or hole~C, 169~m away from the transmitter.  We took data at 10~ft.~(3~m). 
  incremental depths in the boreholes.  The setup used to measure transmission between 
  holes~A~and~B was identical to the setup used to measure between holes~A~and~C 
  so that we could make reliable relative measurements without requiring absolute system 
  calibration.
  
  \subsection{Attenuation measurements}
  The received voltage~($V_{Rx}$) in the far field is given by
  \begin{equation}
    V_{Rx} = \kappa\frac{e^{-\alpha d}}{d}
  \end{equation}
  where~$\alpha$~is the field attenuation coefficient, $d$~is the distance between the 
  transmitting and receiving antennas, and $\kappa$ is a scale factor that accounts for 
  system losses.  Since the system we used was the same regardless of the location of the 
  receiver, we can compare the received voltage at different distances to isolate the 
  attenuation length ($\alpha^{-1}$):
  \begin{equation}
    \alpha^{-1}=\frac{d_2-d_1}{ln(\frac{d_1V_1}{d_2V_2})}.
  \end{equation}
  Figure~\ref{fig:timedomain} shows an example waveform of the received pulse at hole~B 
  and hole~C.
  

  \subsection{Index of refraction measurements}
  \subsection{Reflection measurements}

  \section{Review of GPR data}

  \section{Conclusions}

  
  % The Appendices part is started with the command \appendix;
  % appendix sections are then done as normal sections
  % \appendix
  
  % \section{}
  % \label{}
  
  \begin{thebibliography}{00}
    
    % \bibitem{label}
    % Text of bibliographic item
    
    % notes:
    % \bibitem{label} \note
    
    % subbibitems:
    % \begin{subbibitems}{label}
    % \bibitem{label1}
    % \bibitem{label2}
    % If there is a note, it should come last:
    % \bibitem{label3} \note
    % \end{subbibitems}
    
  \bibitem{gzk}K. Greisen;
    G. Zatsepin, V. Kuz'min.
  \bibitem{ess}R. Engel, D. Seckel, and T. Stanev
  \bibitem{barwick}S. Barwick
  \bibitem{askaryan}Askaryan
  \bibitem{saltzberg}D. Saltzberg SLAC
  \bibitem{hockley}P. Gorham, D. Saltzber et. al.
  \bibitem{halbouty}M. Halbouty, Salt Domes: Gulf Region, United States and Mexico, Gulf Publishing Co., Houston, TX, 1967.
  \bibitem{cherryelog}M. Cherry
  \bibitem{unterberger}Stewart and Unterberger
  \bibitem{firstvisit}A. Connolly, A. Goodhue, M. Cherry, and J. Marsh
  \bibitem{secondvisit}A. Connolly, A. Goodhue, D. Saltzberg
    
  \end{thebibliography}
  
\end{document}

